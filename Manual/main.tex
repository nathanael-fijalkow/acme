\documentclass[11pt]{llncs}

\usepackage[english]{babel}
\usepackage[T1]{fontenc}
\usepackage[latin1]{inputenc}
\usepackage{times}
\usepackage{amssymb,amsmath,amscd,latexsym,graphicx}
\usepackage{framed}
\usepackage[]{algorithm2e}

\newcommand{\set}[1]{\{ #1 \}}
\newcommand{\N}{\mathbb{N}}
\newcommand{\A}{\mathcal{A}}
\newcommand{\M}{\mathcal{M}}
\newcommand{\perm}[1]{\langle #1 \rangle}
\newcommand{\ic}{\mathbf{ic}}
\newcommand{\e}{\mathbf{\varepsilon}}
\newcommand{\re}{\mathbf{r}}
\newcommand{\val}[1]{\mathrm{val}(#1)}
\newcommand{\sem}[1]{[\![#1]\!]}


\newcommand{\PSPACE}{\mathrm{PSPACE}}

\def\mynote#1{{\sf $\dagger$ #1 $\dagger$}}

\title{ACME: Automata with Counters,\\
Monoids and Equivalence}

\author{Nathana\"el Fijalkow\inst{1,2} \and Denis Kuperberg\inst{2}}
\institute{LIAFA, Paris 7 \and University of Warsaw}

\begin{document}

\maketitle

\section{Installation}

\noindent\textbf{Linux}: type ``make linux'' to compile, it produces the executable acme.

\vskip1em
\noindent\textbf{Windows}: you need the make utility package, which you can get here: 
\begin{center}http://gnuwin32.sourceforge.net/packages/make.htm\end{center}
Type ``make win'' to compile, it produces acme.exe.

\section{Input Files}

We describe it line by line:
\begin{itemize}
	\item the first line is the size of the automaton
	\item the second line is the type of the automaton:
 	\begin{itemize}
 		\item c: classical
 		\item p: probabilistic
 		\item n (int): a B-automaton with n counters
 	\end{itemize}
	\item the third line is the alphabet. Each character is a letter, they should not be separated
	\item the fourth line is the initial states. Each state should be separated by spaces
	\item the fifth line is the final states. Each state should be separated by spaces
	\item the next lines are the transition matrices, one for each letter in the input order.
A transition matrix is given by actions (like IE,RI, OO, \_\_) separated by spaces.
Each matrix is preceded by a single character line, the letter (for readability and checking purposes).
 
For classical automata (without counters), the transition matrix contains only 1 and \_.
\end{itemize} 

The blank lines and lines starting with "\%" are ignored.

\section{Available Functions}

\subsection{Computing the Stabilization Monoid of a $B$-Automaton}

\noindent\textbf{In text verbose mode}:
\begin{center}acme -sm Examples/test\_sm.txt -text\end{center}

\noindent\textbf{To visualize it using Graphviz}:
\begin{center}acme -sm Examples/test\_sm.txt -dotty\end{center}

\subsection{Checking the Equivalence of two $B$-Automata}

\noindent\textbf{In text verbose mode}:
\begin{center}acme -equ Examples/test\_equ.txt -text\end{center}

\noindent\textbf{To visualize them using Graphviz}:
\begin{center}acme -equ Examples/test\_equ.txt -dotty\end{center}

\subsection{Running the Markov Monoid Algorithm on a Probabilistic Automaton}

\noindent\textbf{In text verbose mode}:
\begin{center}acme -mma Examples/test\_mma.txt -text\end{center}

\noindent\textbf{To visualize it using Graphviz}:
\begin{center}acme -mma Examples/test\_mma.txt -dotty\end{center}

\subsection{Checking whether a Classical Automaton has the Finite Power Property}

\noindent\textbf{In text verbose mode}:
\begin{center}acme -fpp Examples/test\_fpp\_true.txt -text\end{center}

\noindent\textbf{To visualize it using Graphviz}:
\begin{center}acme -fpp Examples/test\_fpp\_true.txt -dotty\end{center}

\subsection{Drawing an automaton using Graphviz}

\begin{center}acme -dotty Examples/test\_sm.txt -text\end{center}

%\bibliographystyle{alpha}
%\bibliography{bib}
\end{document}
